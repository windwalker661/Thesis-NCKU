%% // nckuee.sty 定義 // cbj

% 產生論文封面
\nckuEEtitlepage
% 產生口試委員會簽名單
\nckuEEoralpage
% 產生口試委員簽名單(en)
%\nckuEEenoralpage

%\newpage
%\setcounter{page}{1}
%\pagenumbering{roman}

%%%%%%%%%%%%%%%%%%%%%%%%%%%%%%%
%       封面內頁
%%%%%%%%%%%%%%%%%%%%%%%%%%%%%%%
% % unmark to add inner cover
%\newpage
%\thispagestyle{empty}
%\thispagestyle{EmptyWaterMarkPage}
%\nckuEEtitlepage


%%%%%%%%%%%%%%%%%%%%%%%%%%%%%%%
%       中文摘要
%%%%%%%%%%%%%%%%%%%%%%%%%%%%%%%

% 可以利用如下自定義的command (定義在nckuee.sty)
% ======
%\begin{zhAbstract}  %中文摘要
%%目的
本研究之目的...
%方法

%結果






\begin{flushleft}
\mbox{{\bf 關鍵字}: 關鍵字、關鍵字、關鍵字、關鍵字}
\end{flushleft}
 % // 可以引入front_cabstract.tex檔案或在此編輯 // cbj
%\end{zhAbstract}

% ...等
% ======

% 在此直接定義如下
%%%%%%%%%%%%%%%%
%
\newpage
% // HongJhe 頁碼起始
\setcounter{page}{1}
\pagenumbering{Roman}
% create an entry in table of contents for 中文摘要
\phantomsection % for hyperref to register this
\addcontentsline{toc}{chapter}{\nameCabstract}
% aligned to the center of the page
\begin{center}
% font size (relative to 12 pt):
% \large (14pt) < \Large (18pt) < \LARGE (20pt) < \huge (24pt)< \Huge (24 pt)
% Set the line spacing to single for the names (to compress the lines)
\renewcommand{\baselinestretch}{1}   %行距 1 倍
% it needs a font size changing command to be effective
\LARGE{\zhTitle}\\  %中文題目
\vspace{0.83cm}
% \makebox is a text box with specified width;
% option s: stretch
% use \makebox to make sure
% each text field occupies the same width
%\makebox[1.5cm][c]{\large{學生:}}
\hspace{0.5in}
\renewcommand{\thefootnote}{\fnsymbol{footnote}}
\makebox[3.5cm][l]{\large{\authorZhName\footnote[1]{}}}\footnotetext[1]{{學生}} % 學生中文姓名
%\hfill
%
%\makebox[3cm][c]{\large{指導教授:}}
\makebox[3.5cm][r]{\large{\advisorZhNameA\footnote[2]{}}}\footnotetext[2]{{指導教授}}  %指導教授1中文姓名
% \hspace{-3em}
\makebox[3.5cm][r]{\large{\advisorZhNameB\footnote[2]{}}}\footnotetext[2]{{指導教授}} \\ %指導教授2中文姓名
%
\vspace{0.42cm}
%
\large{\zhUniv}\large{\zhDepartmentName}\\ %校名系所名
\vspace{0.83cm}
%\vfill
\makebox[2.7cm][c]{\large{摘要}}
\end{center}
% Resume the line spacing to the desired setting
\renewcommand{\baselinestretch}{\mybaselinestretch}   %恢復原設定
%it needs a font size changing command to be effective
% restore the font size to normal
\normalsize
%%%%%%%%%%%%%
\par  % 摘要首段空格 by SianJhe
%目的
本研究之目的...
%方法

%結果






\begin{flushleft}
\mbox{{\bf 關鍵字}: 關鍵字、關鍵字、關鍵字、關鍵字}
\end{flushleft}
 % // 可以引入front_eabstract.tex檔案或在此編輯 // cbj



%%%%%%%%%%%%%%%%%%%%%%%%%%%%%%%
%       英文摘要
%%%%%%%%%%%%%%%%%%%%%%%%%%%%%%%
%
%[method 1]

% 可以利用如下自定義的command (定義在nckuee.sty)
% ======
%\begin{enAbstract}  %英文摘要
%%%%%%%%%%%%%%%%%%%%%%%%%%%%%%%%%%%%%%%%%%%%%%%%%%%%%%%%%%%%%%%%%%%%%%%%%%%%%%%%%%%%
% 以下節錄自《國立成功大學各系(所)博碩士撰寫畢業論文須知補充說明》
% 本案配合98學年度第1學期教務會議決議,未來博士論文3至5年內達成全面化以英文撰寫;10年內達成碩士論文全面化以英文撰寫。  
% 本案為階段性規定,在達成碩博士畢業論文全面英文化之前,以中文撰寫者,須另加附
% 800至1200字之英文延伸摘要。此延伸摘要取代原規定之一頁英文摘要。
% 2021/07/27 新增英文延伸摘要格式
%%%%%%%%%%%%%%%%%%%%%%%%%%%%%%%%%%%%%%%%%%%%%%%%%%%%%%%%%%%%%%%%%%%%%%%%%%%%%%%%%%%

\renewcommand{\baselinestretch}{1}
\begin{center}
    \fbox{%
        \parbox[c]{0.97\textwidth}{%
            \begin{center}
            \normalsize{\bf SUMMARY}\\
            \end{center}
            \normalsize{%英文摘要
            Write down your summary of this paper here...
            %英文關鍵字
            \begin{flushleft}
            \mbox{{\bf Keywords}:Keyword1, Keyword2, Keyword3, Keyword4}
            \end{flushleft}
            }
        }%
    }%
\end{center}
%%%%%%%%%%%%%%%%%%%%%%%%%%%%%%%%%%%%%%%%%%%%%%%%%%%%%%%%%%%%%%%%%%%%%%%%%%%%%%%%%%%
%                                       緒論                                      %
%%%%%%%%%%%%%%%%%%%%%%%%%%%%%%%%%%%%%%%%%%%%%%%%%%%%%%%%%%%%%%%%%%%%%%%%%%%%%%%%%%%
\newpage
\begin{center}
    \textbf{INTRODUCTION}
\end{center}
% 英文緒論
The study on...


%%%%%%%%%%%%%%%%%%%%%%%%%%%%%%%%%%%%%%%%%%%%%%%%%%%%%%%%%%%%%%%%%%%%%%%%%%%%%%%%%%%
%                                       方法                                      %
%%%%%%%%%%%%%%%%%%%%%%%%%%%%%%%%%%%%%%%%%%%%%%%%%%%%%%%%%%%%%%%%%%%%%%%%%%%%%%%%%%%
% \newpage
\begin{center}
    \textbf{MATERIALS AND METHODS}
\end{center}
% 英文研究方法
There are...
%%%%%%%%%%%%%%%%%%%%%%%%%%%%%%%%%%%%%%%%%%%%%%%%%%%%%%%%%%%%%%%%%%%%%%%%%%%%%%%%%%%
%                                    結果與討論                                   %
%%%%%%%%%%%%%%%%%%%%%%%%%%%%%%%%%%%%%%%%%%%%%%%%%%%%%%%%%%%%%%%%%%%%%%%%%%%%%%%%%%%
% \newpage
\begin{center}
    \textbf{RESULTS AND DISCUSSION}
\end{center}
% 英文結果與討論
The results...
%%%%%%%%%%%%%%%%%%%%%%%%%%%%%%%%%%%%%%%%%%%%%%%%%%%%%%%%%%%%%%%%%%%%%%%%%%%%%%%%%%%
%                                       結論                                      %
%%%%%%%%%%%%%%%%%%%%%%%%%%%%%%%%%%%%%%%%%%%%%%%%%%%%%%%%%%%%%%%%%%%%%%%%%%%%%%%%%%%
% \newpage
\begin{center}
    \textbf{CONCLUSION}
\end{center}
% 英文結論
The topics discussed... % // 可以引入front_eabstract.tex檔案或在此編輯 // cbj
%\end{enAbstract}

%[method 2]
\newpage
% create an entry in table of contents for 英文摘要
\phantomsection % for hyperref to register this
\addcontentsline{toc}{chapter}{\nameEabstract} % // HongJhe marked

% aligned to the center of the page
\begin{center}
% font size:
% \large (14pt) < \Large (18pt) < \LARGE (20pt) < \huge (24pt)< \Huge (24 pt)
% Set the line spacing to single for the names (to compress the lines)
\renewcommand{\baselinestretch}{1}   %行距 1 倍
%\large % it needs a font size changing command to be effective
%\large{\nameEabstractc}\\
\large{\bf \enTitle}\\  %英文題目
%\vspace{0.3cm}
% \makebox is a text box with specified width;
% option s: stretch
% use \makebox to make sure
% each text field occupies the same width
%\makebox[2cm][s]{\large{Student: }}
%\hspace{0.3in}
\renewcommand{\thefootnote}{\fnsymbol{footnote}}
\makebox[2cm][c]{\normalsize{Author:}}
\hspace{-0.4in}
\makebox[5cm][c]{\normalsize{\authorEnName\footnote[1]{}}}\footnotetext[1]{{Student}}\\ % 學生英文姓名
\makebox[2cm][c]{\normalsize{Advisor: }}
\hspace{-0.4in}
\makebox[5cm][c]{\normalsize{\advisorEnNameA\footnote[2]{}}}\footnotetext[2]{{Advisor}}  %教授1英文姓名
\hspace{-0.6in}
\makebox[5cm][c]{\normalsize{\advisorEnNameB\footnote[2]{}}}\footnotetext[2]{{Advisor}} \\ %教授2英文姓名
%
%\vspace{0.42cm}
\large{\enDepartmentName}\\ %英文系所全名
%
\large{\enUniv}\\  %英文校名
%\vspace{0.83cm}
%\vfill
%
%\vspace{0.5cm}
\end{center}

% Resume the line spacing the desired setting
\renewcommand{\baselinestretch}{\mybaselinestretch}   %恢復原設定
%\large %it needs a font size changing command to be effective
% restore the font size to normal
\normalsize
%%%%%%%%%%%%%
%%%%%%%%%%%%%%%%%%%%%%%%%%%%%%%%%%%%%%%%%%%%%%%%%%%%%%%%%%%%%%%%%%%%%%%%%%%%%%%%%%%
% 以下節錄自《國立成功大學各系(所)博碩士撰寫畢業論文須知補充說明》
% 本案配合98學年度第1學期教務會議決議,未來博士論文3至5年內達成全面化以英文撰寫;10年內達成碩士論文全面化以英文撰寫。  
% 本案為階段性規定,在達成碩博士畢業論文全面英文化之前,以中文撰寫者,須另加附
% 800至1200字之英文延伸摘要。此延伸摘要取代原規定之一頁英文摘要。
% 2021/07/27 新增英文延伸摘要格式
%%%%%%%%%%%%%%%%%%%%%%%%%%%%%%%%%%%%%%%%%%%%%%%%%%%%%%%%%%%%%%%%%%%%%%%%%%%%%%%%%%%

\renewcommand{\baselinestretch}{1}
\begin{center}
    \fbox{%
        \parbox[c]{0.97\textwidth}{%
            \begin{center}
            \normalsize{\bf SUMMARY}\\
            \end{center}
            \normalsize{%英文摘要
            Write down your summary of this paper here...
            %英文關鍵字
            \begin{flushleft}
            \mbox{{\bf Keywords}:Keyword1, Keyword2, Keyword3, Keyword4}
            \end{flushleft}
            }
        }%
    }%
\end{center}
%%%%%%%%%%%%%%%%%%%%%%%%%%%%%%%%%%%%%%%%%%%%%%%%%%%%%%%%%%%%%%%%%%%%%%%%%%%%%%%%%%%
%                                       緒論                                      %
%%%%%%%%%%%%%%%%%%%%%%%%%%%%%%%%%%%%%%%%%%%%%%%%%%%%%%%%%%%%%%%%%%%%%%%%%%%%%%%%%%%
\newpage
\begin{center}
    \textbf{INTRODUCTION}
\end{center}
% 英文緒論
The study on...


%%%%%%%%%%%%%%%%%%%%%%%%%%%%%%%%%%%%%%%%%%%%%%%%%%%%%%%%%%%%%%%%%%%%%%%%%%%%%%%%%%%
%                                       方法                                      %
%%%%%%%%%%%%%%%%%%%%%%%%%%%%%%%%%%%%%%%%%%%%%%%%%%%%%%%%%%%%%%%%%%%%%%%%%%%%%%%%%%%
% \newpage
\begin{center}
    \textbf{MATERIALS AND METHODS}
\end{center}
% 英文研究方法
There are...
%%%%%%%%%%%%%%%%%%%%%%%%%%%%%%%%%%%%%%%%%%%%%%%%%%%%%%%%%%%%%%%%%%%%%%%%%%%%%%%%%%%
%                                    結果與討論                                   %
%%%%%%%%%%%%%%%%%%%%%%%%%%%%%%%%%%%%%%%%%%%%%%%%%%%%%%%%%%%%%%%%%%%%%%%%%%%%%%%%%%%
% \newpage
\begin{center}
    \textbf{RESULTS AND DISCUSSION}
\end{center}
% 英文結果與討論
The results...
%%%%%%%%%%%%%%%%%%%%%%%%%%%%%%%%%%%%%%%%%%%%%%%%%%%%%%%%%%%%%%%%%%%%%%%%%%%%%%%%%%%
%                                       結論                                      %
%%%%%%%%%%%%%%%%%%%%%%%%%%%%%%%%%%%%%%%%%%%%%%%%%%%%%%%%%%%%%%%%%%%%%%%%%%%%%%%%%%%
% \newpage
\begin{center}
    \textbf{CONCLUSION}
\end{center}
% 英文結論
The topics discussed... % // 可以引入front_eabstract.tex檔案或在此編輯 // cbj

\renewcommand{\baselinestretch}{\mybaselinestretch} 

%%%%%%%%%%%%%%%%%%%%%%%%%%%%%%%
%       誌謝
%%%%%%%%%%%%%%%%%%%%%%%%%%%%%%%
%
% Acknowledgment
\newpage
\phantomsection % for hyperref to register this
%\addcontentsline{toc}{chapter}{\nameAcknc}

\begin{zhAckn}  %誌謝
在這裡寫下你的致謝詞吧。

\begin{flushright}
\mbox{研究生 XXX}
\end{flushright} % // 可以引入front_ackn.tex檔案或在此編輯 // cbj
\end{zhAckn}

%\chapter*{\nameAckn} %\makebox{} is fragile; need protect
%在這裡寫下你的致謝詞吧。

\begin{flushright}
\mbox{研究生 XXX}
\end{flushright} % // 可以引入my_ackn.tex檔案或在此編輯 // cbj
%%testjsjtoejiojsoijtoijos

%%%%%%%%%%%%%%%%%%%%%%%%%%%%%%%
%       目錄
%%%%%%%%%%%%%%%%%%%%%%%%%%%%%%%
%
% Table of contents
\newpage
\renewcommand{\contentsname}{\nameToc}
%\makebox{} is fragile; need protect
\phantomsection % for hyperref to register this
\addcontentsline{toc}{chapter}{\nameTocc}
\tableofcontents
\titlecontents{chapter}
[0em]
{}
{第\CJKnumber{\thecontentslabel}章~~}
{}{\titlerule*{.}\contentspage}
%\titlecontents{section}
%\pgfplotsset{compat=1.17} 
%%%%%%%%%%%%%%%%%%%%%%%%%%%%%%%
%       表目錄
%%%%%%%%%%%%%%%%%%%%%%%%%%%%%%%
%
% List of Tables
\newpage
\renewcommand{\listtablename}{\nameLot}
\newcommand{\lotlabel}{表} \renewcommand{\numberline}[1]{\lotlabel~#1\hspace*{1em}}
%   目錄加入圖、表字樣在前
%   ref:https://minsky.pixnet.net/blog/post/26804386
%\makebox{} is fragile; need protect
\phantomsection % for hyperref to register this
\addcontentsline{toc}{chapter}{\nameLotc}

\listoftables

%%%%%%%%%%%%%%%%%%%%%%%%%%%%%%%
%       圖目錄
%%%%%%%%%%%%%%%%%%%%%%%%%%%%%%%
%
% List of Figures
\newpage
\renewcommand{\listfigurename}{\nameTof}
\newcommand{\loflabel}{圖}
\renewcommand{\numberline}[1]{\loflabel~#1\hspace*{1em}}
%   目錄加入圖、表字樣在前
%   ref:https://minsky.pixnet.net/blog/post/26804386
%\makebox{} is fragile; need protect
\phantomsection % for hyperref to register this
\addcontentsline{toc}{chapter}{\nameTofc}
\listoffigures
%%%%%%%%%%%%%%%%%%%%%%%%%%%%%%%
%       符號說明
%%%%%%%%%%%%%%%%%%%%%%%%%%%%%%%
%
% Symbol list
% define new environment, based on standard description environment
% adapted from p.60~64, <<The LaTeX Companion>>, 1994, ISBN 0-201-54199-8

\newcommand{\SymEntryLabel}[1]%
  {\makebox[3cm][l]{#1}}
%%
\newenvironment{SymEntry}
   {\begin{list}{}%
       {\renewcommand{\makelabel}{\SymEntryLabel}%
        \setlength{\labelwidth}{3cm}%
        \setlength{\leftmargin}{\labelwidth}%
        }%
   }%
   {\end{list}}
%%%
\newpage
\chapter*{\nameSlist} %\makebox{} is fragile; need protect
\phantomsection % for hyperref to register this
\addcontentsline{toc}{chapter}{\nameSlistc}
%
% this file is encoded in utf-8
% v2.0 (Apr. 5, 2009)
%  各符號以 \item[] 包住,然後接著寫說明
% 如果符號是數學符號,應以數學模式$$表示,以取得正確的字體
% 如果符號本身帶有方括號,則此符號可以用大括號 {} 包住保護
\begin{SymEntry}
\item[a]
入射波之波幅
\item[$\bigtriangledown$]
排水體積
\end{SymEntry}

\newpage
\setcounter{page}{1}
\pagenumbering{arabic}
